
%%%%%%%%%%%%%%%%%%%%%%%%%%%%%%%%%%%%%%%%
% 1. Define Keywords, JEL
%%%%%%%%%%%%%%%%%%%%%%%%%%%%%%%%%%%%%%%%
\newcommand{\PAPERKEYWORDS}{\textbf{Keywords}: AI-enabled computer-aid diagnosis, Diagnosis, Skin Sancer, Skin Lesion Classification, Artificial Intelligence, Deep Learning, Machine Learning}
\newcommand{\PAPERJEL}{\textbf{JEL}: I15, D8, D9, O15}

%%%%%%%%%%%%%%%%%%%%%%%%%%%%%%%%%%%%%%%%
% 2. Define Title
%%%%%%%%%%%%%%%%%%%%%%%%%%%%%%%%%%%%%%%%
\newcommand{\PAPERTITLE}{DIAGNOSTIC SUPPORT OF SKIN LESION CLASSIFICATION USING CNN AND SOFT ATTENTION}

%%%%%%%%%%%%%%%%%%%%%%%%%%%%%%%%%%%%%%%%
% 3. Define Authors contact information
%%%%%%%%%%%%%%%%%%%%%%%%%%%%%%%%%%%%%%%%
\newcommand{\AUTHORWANG}{Hoang Khoi Do \\ khoi.dh200322@sis.hust.edu.vn \\ Ha Noi University of Science and Technology}
%\newcommand{\AUTHORWANGINFO}{\AUTHORWANG: Ha Noi University of Science and Technology (email: khoi.dh200322@sis.hust.edu.vn)}
\newcommand{\AUTHOREMAIL}{khoi.dh200322@sis.hust.edu.vn}

%%%%%%%%%%%%%%%%%%%%%%%%%%%%%%%%%%%%%%%%
% 4. Define Thanks
%%%%%%%%%%%%%%%%%%%%%%%%%%%%%%%%%%%%%%%%
\newcommand{\ACKNOWLEDGMENTS}{
We thank \blindtext}

%%%%%%%%%%%%%%%%%%%%%%%%%%%%%%%%%%%%%%%%
% 5. Define Abstract
%%%%%%%%%%%%%%%%%%%%%%%%%%%%%%%%%%%%%%%%
\newcommand{\PAPERABSTRACT}{
Today, the rapid development of industrial zones leads to an increasing incidence of skin diseases because of polluted air. According to a report by the American Cancer Society, it is estimated that in 2022 there will be about 100,000 people suffered from skin cancer and more than 7600 of these people will not survive. In the context that doctors at provincial hospitals and health facilities are overloaded, doctors at lower levels have lack of experience, having a tool to support doctors in the process of diagnosing skin diseases quickly and accurately is essential. Along with the strong development of artificial intelligence technologies, many solutions and tools to support the diagnosis of skin diseases have been researched and developed. These include DenseNet, InceptionNet, ResNet, NasNet, SeNet, EfficientNet, VGGNet. In this study, another approach to build tools to aid in the diagnosis of skin pathologies is proposed. SOTA (state of the art) models DenseNet, InceptionNet, ResNet, NasNet, MobileNet combined with Soft-Attention are used as backbone. In addition, personal information such as age and gender are also used. In addition, a new loss function that takes into account the imbalance of the data is also proposed. Experimental results on dataset HAM10000 show that using InceptionResNetV2 with Soft-Attetion and new loss function gives 90 percent accuracy, mean of precision, f1-score, recall-score, and auc- scores of 0.81, 0.81, 0.82, and 0.989, respectively, are improvements compared to published indexes. Besides, using MobileNetV3Large combined with Soft-Attention and new loss function, even though the number of parameters is 11 times less, the number of floors is 4 times less, it achieves 86 percent accuracy and 30 times faster diagnosis.
}
