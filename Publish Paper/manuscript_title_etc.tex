
%%%%%%%%%%%%%%%%%%%%%%%%%%%%%%%%%%%%%%%%
% 1. Define Keywords, JEL
%%%%%%%%%%%%%%%%%%%%%%%%%%%%%%%%%%%%%%%%
\newcommand{\PAPERKEYWORDS}{\textbf{Keywords}: AI-enabled computer-aid diagnosis, Diagnosis, Skin Sancer, Skin Lesion Classification, Artificial Intelligence, Deep Learning, Machine Learning}
\newcommand{\PAPERJEL}{\textbf{JEL}: I15, D8, D9, O15}

%%%%%%%%%%%%%%%%%%%%%%%%%%%%%%%%%%%%%%%%
% 2. Define Title
%%%%%%%%%%%%%%%%%%%%%%%%%%%%%%%%%%%%%%%%
\newcommand{\PAPERTITLE}{Balanced and Optimized Skin Cancer Classification Model using Soft Attention and Metadata}

%%%%%%%%%%%%%%%%%%%%%%%%%%%%%%%%%%%%%%%%
% 3. Define Authors contact information
%%%%%%%%%%%%%%%%%%%%%%%%%%%%%%%%%%%%%%%%
\newcommand{\AUTHORWANG}{Hoang Khoi Do \\ khoi.dh200322@sis.hust.edu.vn \\ Ha Noi University of Science and Technology}
%\newcommand{\AUTHORWANGINFO}{\AUTHORWANG: Ha Noi University of Science and Technology (email: khoi.dh200322@sis.hust.edu.vn)}
\newcommand{\AUTHOREMAIL}{khoi.dh200322@sis.hust.edu.vn}

%%%%%%%%%%%%%%%%%%%%%%%%%%%%%%%%%%%%%%%%
% 4. Define Thanks
%%%%%%%%%%%%%%%%%%%%%%%%%%%%%%%%%%%%%%%%
\newcommand{\ACKNOWLEDGMENTS}{
We thank \blindtext}

%%%%%%%%%%%%%%%%%%%%%%%%%%%%%%%%%%%%%%%%
% 5. Define Abstract
%%%%%%%%%%%%%%%%%%%%%%%%%%%%%%%%%%%%%%%%
\newcommand{\PAPERABSTRACT}{
Today, the rapid development of big cities and industrial zones leads to an increasing incidence of skin diseases because of air pollution. According to a survey by the American Cancer Society, there are about 100,000 people with skin cancer and more than 7,600 people are predicted to not survive by 2022. Therefore, a tool to support the diagnosis of skin cancer quickly and accurately is needed in the context of hospitals being overwhelmed every day by the large number of patients in many developing countries. In the past half-decade, technologies belonging to the 4.0 technology group are rapidly changing our lives, including Artificial Intelligence, Machine Learning and Deep Learning. Deep Learning and Machine Learning have also been applied to build diagnostic models of skin pathologies. Some of the new approaches are GradCam, Kernel Shap and the Student-Teacher model. In Machine Learning, Random Forest and Support Vector Machine are applied. The data used in this study is HAM10000 which is an imbalanced dataset. Several methods have been applied including data augmentation, Soft-Attention, and balanced resampling. In this study, the models DenseNet201, InceptionResNetV2, ResNet50, ResNet152, NasNetLarge are combined with Soft Attetion by replacing the last 2 CNN blocks with Soft-Attetnion. Moreover, metadata is used as a sub input layer in the model. Besides, weight loss function is applied, the way of calculating weight will be shown in this paper. Soft Attention was tested in an earlier paper to improve the model's performance with 92 percent accuracy. In addition, MobileNetV2, MobileNetV3 and NasNetMobile models are also studied to deploy solutions on phones. After testing, the performance of DenseNet201 model combined with Soft-Attetion using baseline data and weighted loss function reached 90 percent accuracy but standard deviation of f1 score and recall score of it is 0.08 and 0.09, respectively, compared to the standard deviation of the f1 index and the recall index of 0.22 and 0.27 of the previous paper, respectively.A model with combination of MobileNetV3Large and Soft-Attention layer with baseline data and image input have been constructed that achieve 86 percent accuracy but training time takes 116 seconds per training session.
}
